\title{Fuzz Testing with Inferred Attribute Grammars}
\author{
        Tim Henderson\\
        Case Western Reserve University\\
}
\date{\today}

\maketitle

\begin{abstract}
Blackbox Fuzzing, a type automated dynamic test generation, has been heavily
used to find security vulnerabilities and faults in software. We propose a new
form of Blackbox Fuzzing: Inferred Attribute Grammar Fuzzing. Attribute grammars
have a long distinguished history in specifying context sensitive restrictions to
context free languages. We use them here to assist with generating correct
randomized inputs for programs with structured input languages.  Attribute
Grammars are time consuming to write by hand, and a grammar for the input
language of the program under test may not be available. Therefore, we
present a multi-step approach to constructing inferred Attribute Grammars useful
for Fuzz Testing. Our approach assumes the existence of a large set of
operational inputs.
\end{abstract}

this is a test


\begin{multicols}{2}

\section{Introduction}

Attribute Grammars were invented in the heady days of the late 1960's by Dr.
Knuth.\cite{Knuth1990} At the time the question on how to formally define the
semantics of a programming language had become a fashionable research topic.
Attribute grammars emerged as a popular way and remain important for such
purposes today. The key to their success is the ability to define context
sensitive semantics in languages. A common example, variable names must be
declared before use.

Input Languages to programs are often context sensitive. The obvious and natural
example is computer programming languages which encompass a wide variety of
useful programs. Web browsers, text editors, integrated development environments,
web servers, and game engines all make use of programming languages. Sometimes
exposing the languages to the end user for extensibility purposes. Other program
types have similarly complex input schemes often arising from natural world,
for example organic chemical structures.

Fuzzing, in particular ``Blackbox Fuzzing'' automatically generates test cases
for software. In the simplest formulation the test cases are little more than
random data. These types of tests produce poor coverage of the target program.
Context free grammar based fuzzing produces better results. However, context
free grammars are still unable to achieve good coverage of the entire target
program. 

Why do context free grammars fail when used to generate test cases for software
even when the grammar is known (and not inferred)? Context free grammars only
capture the syntax of the language and not the semantics. This means, the
language the program accepts as input is strictly a subset of the language the
CFG generates. Depending on the size of the gap between the CFG and the actual
language a random string generated from the CFG may only have a very small
chance of being in the language.

\section{Using Attribute Grammars to Generate Test Cases}

Attribute grammars can be used in a straight forward way to generate test cases. 

Example Grammar Here

Example Derivation

Generation Algorithm

\section{Annotating Context Free Grammars with Attributes}

Context free grammars can be annotated with synthesized attributes using a
straight forward process if the types of attributes are reasonably restricted.
For this paper we allow three types of attributes: 1) production choice, eg.
which production for the non-terminal was chosen, 2) scalar attributes, eg. the
value of a token or the name of a token, 3) set attributes, eg. collections of
scalar attributes. 

why the restrictions are reasonable 

exact bound on what can be expressed

\subsection{Algorithm Sketch}

All possible attributes are synthesized

Conditions are proposed based, and tested against operational inputs. Only those
conditions which pass all operational inputs are kept.

Attributes un-used by final conditions are pruned.

\section{Conclusion}


\nocite{*}
\bibliographystyle{acm}
\bibliography{bibliography}

\end{multicols}

