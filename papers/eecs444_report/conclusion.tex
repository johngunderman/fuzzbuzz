\section{Future Work} \subsection{Attribute Grammar} future work blah
\subsubsection{Annotating Context Free Grammars with Attributes}

Context free grammars can be annotated with synthesized attributes using a
straight forward process if the types of attributes are reasonably restricted.
For this paper we allow three types of attributes: 1) production choice, eg.
which production for the non-terminal was chosen, 2) scalar attributes, eg. the
value of a token or the name of a token, 3) set attributes, eg. collections of
scalar attributes. 

why the restrictions are reasonable 

exact bound on what can be expressed

\paragraph{Algorithm Sketch}

All possible attributes are synthesized

Conditions are proposed based, and tested against operational inputs. Only those
conditions which pass all operational inputs are kept.

Attributes un-used by final conditions are pruned.

\subsection{CFG Stats}

\subsection{Mutation Fuzzing} The current implementation of mutation fuzzing in
Fuzzbuzz is a great foundation for future work. There is a clear path to
integrating CFG stats and attribute grammars into the generation process. By
integrating these methods, it is likely that Fuzzbuzz will be able to prune the
mutation space to sufficient level to generate mutations that are effective at
fault discovery.

\section{Conclusion}

The goal of the Fuzzbuzz project is to create a corpus driven blackbox fuzzing
framework for structured languages. In the work covered by this paper we
assembled several critical pieces of the system but we have not yet arrived at a
complete system.

