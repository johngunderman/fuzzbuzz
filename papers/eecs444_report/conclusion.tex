\section{Conclusion}

The goal of the Fuzzbuzz project is to create a corpus driven blackbox fuzzing
framework for structured languages. In the work covered by this paper we
assembled several critical pieces of the system but we have not yet arrived at a
complete system. To review we achieved the following major results: 1) proved
the complexity class of generating strings from attribute grammars, 2) created a
``first draft'' algorithm for generating strings from attribute grammars, 3)
verified the usefulness of conditional probabilities for generating strings
based on context free grammars, and 4) created a ``first draft'' mutation
fuzzer.

\subsection{Future Work}

There are several key tasks to be completed in order to create a fully
functional fuzzer. First is to integrated statistics driven rule choice in
the attribute grammar string generation. This work was prototyped use context
free grammars and is now ready for integration into the core. Once integrated
the attribute grammar string generator can be tested on more complex grammars.
Second, the attribute grammar string generator needs to be integrated into the
mutation fuzzer. Then, when examples are mutated they will continue to obey the
semantic rules in the grammar. Currently strings generated by the mutation
fuzzer are often nonsensical, if the engine enforced the semantic constraints
the results could be improved. Third, test the system on a real system such as a
compiler to determine if it works as expected.

If the system seems to work well with hand written attribute grammars then it
seems it would be worthwhile to attempt to inference the attribute from
examples. This would have several benefits: 1) Attributes would no longer have
to be hand written and 2) Attributes would reflect the input corpus potentially
being more constrained than hand written attributes. 

