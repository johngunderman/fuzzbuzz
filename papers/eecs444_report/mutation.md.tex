\section{Mutation Fuzzing with CFG Stats and Attribute Grammars}

Mutation fuzzing is a useful category of fuzzing in its ability to
generate output that is similar to existing samples. These outputs can
then be used as inputs to a program for the purpose of detecting
faults.

Mutation fuzzing is often used to generate a large set of mutants
surrounding a singular example. For practical purposes, these mutant
sets are pruned by use of various heuristics, to obtain a mutant set
which can be executed in a reasonable amount of time.

Jia and Harman in their 2010 paper survey the general techniques used
for pruning the mutant set.\cite{Jia2010} These techniques evaluate
mutations based on what tests ``kill'' them. A test T is said to kill
a mutation M if for source S, T(S) does not give the same output as
T(M). If S and M match on T, then M is said to have ``survived''. This
evaluation method unfortunately has one serious flaw, in the form of
an undecidable problem. In some instances, mutants may be generated
that are semantically the same, but syntactically different. Due to
the undecidability of program equivalence, it is impossible to detect
these cases. Jia recognizes this problem as being a major hindering
factor in the use of mutation fuzzing. A secondary main factor is the
need for human interaction, both as a human oracle and as a mediator
for the mutant equivalence problem.

In Jia's paper, four seperate techniques for mutation set pruning are
discussed. The simplest of their surveyed techniques is to simply to
pick from the set at random. While effective at pruning, this
technique clearly lacks any ability to select the best mutants from
the set. As a step up from random selection, clustering has been
proposed as an improved pruning method. Using clustering, samples can
be selected from each cluster in order to maintain a wide range of
mutations. This clustering process is implemented by grouping
mutations that are killed by the same set of tests.
% TODO: Possibly go into Hussain's master thesis here?

The two remaining techniques attempt to use more complicated methods
to prune. The first of these, selective mutation, constrains the number
of mutations. Some mutation operations may produce excessive or
redundant mutants. By eliminating these mutations, the generated set
of mutants starts out small, largely eliminating the need for further
reductions.

The fourth and final technique discussed  by Jia and Harman is that of
Higher Order Mutation (HOM). As opposed to most mutations, which fall
into the category of First Order Mutations (FOM), HOMs are generated
by applying mutation operators multiple times.


Current research in the area spans a wide variety of techniques. The
simplest of these techniques is first order mutations (FOM)



% ------------------------------

In Fuzzbuzz, we decided to develop mutation fuzzing as a compliment to
the CFG stat and attribute grammar methods. Usage of these three
techniques in conjunction is not a widely explored area. We predict
that these techniques will better prune the mutation space as compared
to other contemporary pruning methods.


