\section{Mutation Fuzzing with CFG Stats and Attribute Grammars}

\subsection{Introduction}
Mutation fuzzing is a useful category of fuzzing in its ability to
generate output that is similar to existing samples. These outputs can
then be used as inputs to a program for the purpose of detecting
faults. Given the huge set of potential faults in a given program,
mutation fuzzing attempts to limit the scope of relevant faults to
those caused by input similar to existing input sets.

Mutation fuzzing is often used to generate a large set of mutants
surrounding a singular example. For practical purposes, these mutant
sets are pruned by use of various heuristics, to obtain a mutant set
which can be executed in a reasonable amount of time.

Mutation fuzzing is a growing field which has been researched
increasingly over the past thirty years. Between 1977 and 2009, there
were no less than 350 papers on the subject, with the majority of the
papers being published in the last decade alone. \cite{Jia2010}

\subsection{Literature Survey}
Jia and Harman in their 2010 paper survey the general techniques used
for pruning the mutant set.\cite{Jia2010} These techniques evaluate
mutations based on what tests ``kill'' them. A test T is said to kill
a mutation M if for source S, T(S) does not give the same output as
T(M). If S and M match on T, then M is said to have ``survived''. This
evaluation method unfortunately has one serious flaw, in the form of
an undecidable problem. In some instances, mutants may be generated
that are semantically the same, but syntactically different. Due to
the undecidibility of program equivalence, it is impossible to detect
these cases. Jia recognizes this problem as being a major hindering
factor in the use of mutation fuzzing. A secondary main factor is the
need for human interaction, both as a human oracle and as a mediator
for the mutant equivalence problem.

In Jia's paper, four separate techniques for mutation set pruning are
discussed. The simplest of their surveyed techniques is to simply to
pick from the set at random. While effective at pruning, this
technique clearly lacks any ability to select the best mutants from
the set. As a step up from random selection, clustering has been
proposed as an improved pruning method. Using clustering, samples can
be selected from each cluster in order to maintain a wide range of
mutations. This clustering process is implemented by grouping
mutations that are killed by the same set of tests.
% TODO: Possibly go into Hussain's master thesis here?

The two remaining techniques attempt to use more complicated methods
to prune. The first of these, selective mutation, constrains the number
of mutations. Some mutation operations may produce excessive or
redundant mutants. By eliminating these mutations, the generated set
of mutants starts out small, largely eliminating the need for further
reductions.

The fourth and final technique discussed  by Jia and Harman is that of
Higher Order Mutation (HOM). As opposed to most mutations, which fall
into the category of First Order Mutations (FOM), HOMs are generated
by applying mutation operators multiple times.

An interesting alternative technique as proposed by Papadakis et
al. is to the control flow graph of the source input to assist in the
generation of mutants. \cite{Papadakis2010} This approach is only
relevant to inputs that are programming languages, as grammars in
general do not necessarily have control flow graphs.

In Papadakis' approach, by selectively mutating the control flow graph
and testing the coverage of the new graph, mutations can be generated
that give better coverage.

The most important point in Papadakis' paper is the reachability of
the new mutations. By transforming the problem of ``killing'' mutants
to the problem of reaching all the mutated branches in a graph, the
generation of valid mutants can be simplified. Papadakis continues,
discussing how generating coverage for a graph is a well researched
problem and provides a viable alternative to standard approaches in
mutation fuzzing.





\subsection{Methods}

In Fuzzbuzz, we decided to develop mutation fuzzing as a compliment to
the CFG stat and attribute grammar methods. Usage of these three
techniques in conjunction is not a widely explored area. We predict
that these techniques will better prune the mutation space as compared
to other contemporary pruning methods.

Our work on mutation fuzzing did not extend as far as we wanted, but
it certainly set a firm foundation for future work. Support was added
to Fuzzbuzz for basic mutation based on the random selection
algorithm. Fuzzbuzz mutates nonterminals inside of the example AST in
order to produce the final result.

The current implementation is split into two easily extensible
functions. The first of these is the selection function. This function
decides what subtree of the AST should be mutated. While a simplistic
algorithm is currently employed, the system allows for the selection
function to be easily swapped out. The second important function is
the generator function. This function generates a subtree for the
given non-terminal. This part in particular is where the CFG stat and
attribute grammar implementations become very useful. While not
currently implemented as such, these two tree generation techniques
can be used for better mutation generation.


